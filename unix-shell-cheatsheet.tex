%%%%%%%%%%%%%%%%%%%%%%%%%%%%%%%%%%%%%%%%%
% Cheatsheet
% LaTeX Template
% Version 1.0 (12/12/15)
%
% This template has been downloaded from:
% http://www.LaTeXTemplates.com
%
% Original author:
% Michael Müller (https://github.com/cmichi/latex-template-collection) with
% extensive modifications by Vel (vel@LaTeXTemplates.com)
%
% License:
% The MIT License (see included LICENSE file)
%
%%%%%%%%%%%%%%%%%%%%%%%%%%%%%%%%%%%%%%%%%

%----------------------------------------------------------------------------------------
%	PACKAGES AND OTHER DOCUMENT CONFIGURATIONS
%----------------------------------------------------------------------------------------

\documentclass[11pt, letterpaper]{scrartcl} % 11pt font size

\usepackage[utf8]{inputenc} % Required for inputting international characters
\usepackage[T1]{fontenc} % Output font encoding for international characters
\usepackage[margin=0pt, landscape]{geometry} % Page margins and orientation
\usepackage{graphicx} % Required for including images
\usepackage{color} % Required for color customization
\definecolor{mygray}{gray}{.75} % Custom color

\usepackage{url} % Required for the \url command to easily display URLs

\usepackage[ % This block contains information used to annotate the PDF
	colorlinks=true, 
	pdftitle={Unix Shell Cheatsheet}, 
	pdfauthor={Oliver Stueker}, 
	pdfsubject={Compilation of useful commands on the Unix/Linux Terminal}, 
	pdfkeywords={Unix, Linux, Terminal, bash, shell, Cheatsheet}
]{hyperref}

% \usepackage{listings}
\usepackage{alltt}
\setlength{\unitlength}{1mm} % Set the length that numerical units are measured in
\setlength{\parindent}{0pt} % Stop paragraph indentation
\renewcommand{\ttdefault}{txtt}

\renewcommand{\dots}{\ \dotfill{}\ } % Fills in the right amount of dots
\newcommand{\command}[2]{\texttt{#1}~\dotfill{}~#2\\} % Custom command for adding a shorcut
\newcommand{\sectiontitle}[1]{\paragraph{#1} \ \vspace{0.2cm} \\} % Custom command for subsection titles
\newcommand{\sectiontitleB}[2]{\paragraph{#1} \ \\ ~#2 \vspace{0.2cm} \\} % Custom command for subsection titles
%\titlespacing*{\sectiontitle}{0pt}{}{4.3ex plus .2ex}

%----------------------------------------------------------------------------------------

\begin{document}

\begin{picture}(279,215) % Create a container for the page content

%----------------------------------------------------------------------------------------
%	TITLE SECTION 
%----------------------------------------------------------------------------------------

\put(5,210){ % Position on the page to put the title
\begin{minipage}[t]{210mm} % The size and alignment of the title
\section*{Unix Reference Sheet} % Title
\end{minipage}
}

%----------------------------------------------------------------------------------------
%	FIRST COLUMN SPECIFICATION
%----------------------------------------------------------------------------------------

\put(5,195){ % Divide the page
\begin{minipage}[t]{85mm} % Create a box to house text

%----------------------------------------------------------------------------------------
%	HEADING: Navigating Directories
%----------------------------------------------------------------------------------------

\sectiontitle{Navigating Directories}
\command{\textbf{ls}}							{\emph{list} content of current directory}
\command{\textbf{ls -a}}						{\emph{list} all (incl. hidden)}
\command{\textbf{ls -A}}						{\emph{list} Almost all (except . and ..)}
\command{\textbf{ls -l}}						{\emph{list} long (permissions, size, time)}
\command{\textbf{cd} \emph{<dir>}}				{\emph{change directory} to  <dir>}
\command{\textbf{cd}}							{go back to HOME directory}
\command{\textbf{mkdir} \emph{<dir>}}			{\emph{make} new \emph{directory}}
\command{\textbf{rmdir} \emph{<dir>}}			{\emph{remove} empty \emph{directory}}
\command{\textbf{pwd}}							{\emph{print working directory}}
\command{\textbf{rm} \emph{<file> [<f2> ...]}}	{\emph{remove} one or more files}
\command{\textbf{rm -R} \emph{<dir>}}			{\emph{remove} <dir> incl. all content}

%----------------------------------------------------------------------------------------
%	HEADING: Working with Files
%----------------------------------------------------------------------------------------				
			
\sectiontitle{Working with Files}\vspace{0.2cm}
\command{\textbf{cp} \emph{<from>  <to>}}	{\emph{copy} file\\
\null\hfill <from> can be several files (wildcard)\\
\null\hfill <to> can be a filename or directory}
\command{\textbf{mv} \emph{<from>  <to>}}	{\emph{move} (or rename) file/dir\\
\null\hfill <from> can be several files (wildcard)}
\command{\textbf{find} \emph{<dir>} -name \emph{name}}	{find files in <dir>}

\sectiontitleB{Working with Text}
{The following commands can take text from standard-input (STDIN), instead of <file>:}
\command{\textbf{cat}  \emph{<file>}}				{show content of file}
\command{\textbf{head} \emph{<file>}}				{show first lines}
\command{\textbf{tail} \emph{<file>}}				{show last lines}
\command{\textbf{less} \emph{<file>}}				{pager (press q to quit)}
\command{\textbf{grep} \emph{<keyw> <file>}}		{filter lines with \emph{<keyw>}}
\command{\textbf{sort} \emph{<file>}}				{sort lines}
\command{\textbf{uniq} \emph{<file>}}				{remove duplicates}
\command{\textbf{sed}  \emph{'s/foo/bar/g' <file>}}	{replace \emph{foo} with \emph{bar}}
% \command{\textbf{awk} }	{}

%----------------------------------------------------------------------------------------
%	HEADING: Convert Line Endings in files
%----------------------------------------------------------------------------------------	
\sectiontitle{Convert Line Endings in files}
\command{\textbf{dos2unix} \emph{<file>}}		{convert from Win to Unix}
\command{\textbf{unix2dos} \emph{<file>}}		{convert from Unix to Win}


%----------------------------------------------------------------------------------------

\end{minipage} % End the first column of text
} % End the first division of the page

%----------------------------------------------------------------------------------------
%	SECOND COLUMN SPECIFICATION 
%----------------------------------------------------------------------------------------

\put(96,195){ % Divide the page
\begin{minipage}[t]{85mm} % Create a box to house text

%----------------------------------------------------------------------------------------
%	HEADING: Redirecting Input / Output
%----------------------------------------------------------------------------------------	
\sectiontitle{Redirecting Input / Output}
\command{\emph{cmd} > \emph{file}}				{redirect output of \emph{cmd} into \emph{file}\\
												\null\hfill this overwrites \emph{file}}
\command{\emph{cmd} >{}> \emph{file}}			{append output of \emph{cmd} to \emph{file}}
\command{\emph{cmd} < \emph{file}}				{\emph{cmd} read input from \emph{file}}
\command{\emph{cmd1} | \emph{cmd2}}				{\emph{pipe}: the output of \emph{cmd1}\\
												\null\hfill becomes the input for \emph{cmd2}}

%----------------------------------------------------------------------------------------
%	HEADING: SSH Secure Shell
%----------------------------------------------------------------------------------------	
\sectiontitle{SSH Secure Shell}
\command{\textbf{ssh} \emph{jsmith@hostname.example.com\\}}			{connect to remote host as user jsmith}
\command{\textbf{scp} \emph{<file> jsmith@host.example.com:data/\\}}	{copy \emph{file} from here to ~/data/ on remote host}
\command{\textbf{scp} \emph{jsmith@host.example.com:data/file ./\\}}	{copy \emph{~/data/file} from remote host to here}
\command{\textbf{sftp} \emph{jsmith@hostname.example.com\\}}			{}

% \command{\textbf{rsync -av }}	{}

%----------------------------------------------------------------------------------------
%	HEADING: line-editing
%----------------------------------------------------------------------------------------	
\sectiontitle{Line editing}
\command{Ctrl + a  \emph{or} [Home]}		{go to start of line}
\command{Ctrl + e  \emph{or} [End]}			{go to end of line}
\command{Ctrl + k}							{\emph{kill} line to end}
\command{Ctrl + u}							{kill line to start}
\command{Ctrl + w}							{delete word backwards}
\command{Ctrl + y}							{\emph{yank} deleted text}

%----------------------------------------------------------------------------------------
%	HEADING: history
%----------------------------------------------------------------------------------------	
\sectiontitle{Working with history}
\command{$\uparrow$}						{flip backwards in history}
\command{$\downarrow$}						{flip forward in history}
\command{Alt + <}							{jump to first entry in history}
\command{Alt + Shift + >}					{jump to prompt (end of hist.)}
\command{Ctrl + r}							{search backwards in history}
\command{Ctrl + g}							{abort history search}
% <command>  !!           run previous command prefixed with <command>


%----------------------------------------------------------------------------------------

\end{minipage} % End the second column of text
} % End the second division of the page

%----------------------------------------------------------------------------------------
%	THIRD COLUMN SPECIFICATION 
%----------------------------------------------------------------------------------------

\put(190,195){ % Divide the page
\begin{minipage}[t]{85mm} % Create a box to house tex
%----------------------------------------------------------------------------------------
%	HEADING: Special Characters
%----------------------------------------------------------------------------------------	

\sectiontitle{Special Characters}
\command{\textbf{.}}						{current directory}
\command{\textbf{..}}						{parent directory}
\command{\textbf{$\sim$}}					{short for the home directory}
\command{\textbf{/}}						{directory separator, e.g. \emph{/home/jsmith}}
\command{\emph{/home/jsmith/data}}			{absolute path starts with /}
\command{\emph{../project} or \emph{./data}}{relative paths}

\sectiontitle{Wildcards}
\command{\textbf{*}}						{all files}
\command{\textbf{*.txt}}					{all files ending in .txt}
\command{\textbf{data*}}					{all files starting with data}

%----------------------------------------------------------------------------------------
%	HEADING: Getting Help
%----------------------------------------------------------------------------------------	

\sectiontitle{Getting Help}
\command{\emph{command} -h}				{most commands show a help}
\command{\emph{command} -{}-help}	{most commands show a help}
\command{man \emph{command}}			{show manual page for command\\
												\null\hfill navigate like less}
\command{man -k  \emph{keyword}}		{list man pages with \emph{keyword}}
\command{apropos \emph{keyword}}		{list man pages with \emph{keyword}}

\vspace{\baselineskip} % Whitespace before the next section

%----------------------------------------------------------------------------------------
%	LINKS AND INFORMATION
%----------------------------------------------------------------------------------------

\sectiontitle{Links and information}
% \url{http://tmuxcheatsheet.com/} \\
% \url{https://leanpub.com/the-tao-of-tmux/read}

%----------------------------------------------------------------------------------------
%	FOOTNOTE
%----------------------------------------------------------------------------------------

\vspace{\baselineskip}
\linethickness{0.5mm} % Thickness of the footer line
{\color{mygray}\line(1,0){30}} % Print the line with a custom color

\footnotesize{
Created by Oliver Stueker, 2017-2021\\ 
Released under the MIT license.\\
\url{https://github.com/ostueker/unix-command-reference}
}

%----------------------------------------------------------------------------------------

\end{minipage} % End the third column of text
} % End the third division of the page
\end{picture} % End the container for the entire page

%----------------------------------------------------------------------------------------

\end{document}
